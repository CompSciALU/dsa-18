\documentclass[pdf]{beamer}
    
    \usepackage[utf8]{inputenc} 
    \usepackage[T1]{fontenc}
    \usepackage[UKenglish]{babel}
    \usepackage{listings}
    \usepackage{hyperref}
    
    \mode<presentation>{\usetheme[numbering=none, progressbar=head, block=fill]{metropolis}}
    \title{Abstract Data Types}
    \subtitle{Week 2}
    \author{Stephen Naicken}
    \institute[ALU]{African Leadership University}
    
\begin{document}
\begin{frame}
    \titlepage
\end{frame}

\section{Updates}

\begin{frame}{Assessment}
    \begin{itemize}
        \item 50\% exam
        \item 50\% coursework
        \begin{itemize}
            \item 60\% from a programming assignment (W1-5).
            \item 20\% from a problem set (W6-7).
            \item 20\% online quizzes on GCU Learn.
        \end{itemize}
        \item Assignment release and submission dates TBC.
    \end{itemize}
\end{frame}

\begin{frame}{Lab Groups}
    Please visit \url{http://bit.ly/2JaAMNT} to find out which lab session you will be required to attend. \textbf{Attendance is compulsory}.
\end{frame}

\section{Week 1: Recap}

\begin{frame}{Bag}
    % \begin{block}{Process}
        In week 1, we implemented a model of a bag of Lego bricks in Java.  Students did the following:
        
        \begin{itemize}
            \item Initial implementation using a single class.
            \item Changed the implementation to use an interface and an array.
        \end{itemize}
        
    % \end{block}
    
    \begin{block}{Questions}
        \begin{itemize}
            \item Why did we make use of an interface?
            \item Why did we switch to an array?
            \item How is this connected to ADTs?
        \end{itemize}
    \end{block}
\end{frame}

\begin{frame}[fragile]{Student code}
    \begin{block}{Bag.java}
    \begin{lstlisting}[language=Java,basicstyle=\ttfamily\tiny,keywordstyle=\color{blue}]
        public class Bag {
            List<LegoBrick> al;
        
            public Bag(){
                al = new ArrayList<LegoBrick>();    
            }
        
            public void addBrick(LegoBrick newBrick){
                al.add(newBrick);
            }
        
            public void removeBrick(){
                if(!al.isEmpty()){
                    Random rand = new Random();
                    int  n = rand.nextInt(al.size());
                    al.remove(n);
                } else {
                    System.out.println("the bag is empty");
                }
            }
        
            public void empty(){
                al.clear();
            }
        }
    \end{lstlisting}
\end{block}
\end{frame}

\begin{frame}{Improving the Bag class}
    \begin{itemize}
        \item<1-2> Our Bag can only hold one type of object, LegoBrick.
        \item<1-2> But we might want a Bag the can hold other types of objects.
        \item<1-2> How can we make this so?
        \item<2> We can use an interface.
    \end{itemize}
\end{frame}

\begin{frame}[fragile]{Bag interface}
    \begin{lstlisting}[language=Java,basicstyle=\ttfamily\tiny,keywordstyle=\color{blue}]
    interface Bag <T>{

        // add an object of type T to the bag
        void add(T t);

        // remove an object of type T from the bag
        T remove(T t);

        // check if an object of type T is contained in the bag
        boolean contains(T t);

        // returns the number of items in the bag
        int size();

        // empty the bag
        T[] empty();

    }
    \end{lstlisting}

\end{frame}

\begin{frame}{Abstract Data Type}
    An Abstract Data Type (ADT) defines:
    \begin{itemize}
        \item<1-3> A well-specified collection of data (data state)
        \item<1-3> A set of operations that can be performed upon the data
    \end{itemize}

    \begin{block}<2-3>{Question}
        Is the Bag interface an ADT?
    \end{block}

    \begin{block}<3>{Answer}
        Yes, but who can explain why?
    \end{block}
\end{frame}

\begin{frame}[fragile]{Bag ADT}
    \begin{block}{A well-specified collection of data (data state)}
        \begin{lstlisting}
    interface Bag <T>{
        \end{lstlisting}
    \end{block}

    A Bag ADT that can hold objects of type T.
\end{frame}

\begin{frame}[fragile]{Bag ADT}
    \begin{block}{A well-specified collection of data (data state)}
        \begin{lstlisting}
    interface Bag <T extends Student>{
        \end{lstlisting}
    \end{block}

    \begin{block}{Question}
        A Bag ADT that can hold objects of type... ?.
    \end{block}
\end{frame}

\begin{frame}[fragile]{Bag ADT}
    \begin{block}{A well-specified collection of data (data state)}
        \begin{lstlisting}
    interface Bag <? extends T>{
        \end{lstlisting}
    \end{block}

    \begin{block}{Question}
        A Bag ADT that can hold objects of type... ?.
    \end{block}
\end{frame}

\begin{frame}[fragile]{Bag ADT}
    \begin{block}{A set of operations that can be performed upon the data}
        \begin{lstlisting}[language=Java,basicstyle=\ttfamily\tiny,keywordstyle=\color{blue}]
            // add an object of type T to the bag
            void add(T t);

            // remove an object of type T from the bag
            T remove(T t);

            // check if an object of type T is contained in the bag
            boolean contains(T t);

            // returns the number of items in the bag
            int size();

            // empty the bag
            T[] empty();
        \end{lstlisting}
    \end{block}
\end{frame}

\begin{frame}{Data Structure}
    \begin{itemize}
        \item A base storage method (e.g. an array, a list, a tree, a …)
        \item One or more algorithms that are used to access or modify that data
    \end{itemize}

    A class that implements the ADT interface is a model of a data structure (Concrete Data Type).
\end{frame}

\begin{frame}[fragile]{BetterLegoBag}
    \begin{block}{A base storage method}
        \begin{lstlisting}[language=Java,basicstyle=\ttfamily\scriptsize,keywordstyle=\color{blue}]
        // backing data structure to store the object
        private Lego[] legos;
        \end{lstlisting}
    \end{block}
\end{frame}

\begin{frame}{LegoBag}
    \begin{block}{One or more algorithms that are used to access or modify that data}
        %\begin{lstlisting}[language=Java,basicstyle=\ttfamily\tiny,keywordstyle=\color{blue}]
            All the methods implementing the methods defined in the interface (i.e. contract).
        %\end{lstlisting}
    \end{block}
\end{frame}

\begin{frame}[fragile]{Polymorphism and Encapsulation}
    \begin{block}{Encapsulation}
        \begin{lstlisting}[language=Java,basicstyle=\ttfamily\scriptsize,keywordstyle=\color{blue}]
    // backing data structure to store the object
    private Lego[] legos;
        \end{lstlisting}

        The internals of the data structure are hidden and only exposed through the methods defined in the interface (and implemented in the class).
    \end{block}
\end{frame}

\begin{frame}[fragile]{Polymorphism and Encapsulation}
    \begin{block}{Polymorphism}
        \begin{lstlisting}[language=Java,basicstyle=\ttfamily\scriptsize,keywordstyle=\color{blue}]
    public class Thief {

        void steal(Bag<?> bag){
            // Thief can steal any bag!
        }
    }
        \end{lstlisting}
    \end{block}
\end{frame}

\section{Week 2}

\begin{frame}{Learning Outcomes}
    \begin{block}{What you must know}
        \begin{itemize}
            \item The ADT definition.
            \item An ADT is abstract and requires an implementation.
            \item The interface is the ADT.
            \item The implementation is a data structure.
            \item Know the rationale/benefits of ADTs.
        \end{itemize}
    \end{block}
\end{frame}

\begin{frame}{Learning Outcomes}
    \begin{block}{What you must be able to do}
        \begin{itemize}
            \item Code a Java interface to define an ADT.
            \item Code an implementation of a simple ADT.
        \end{itemize}
    \end{block}
\end{frame}

\begin{frame}{Tasks \& Deliverables}
    \begin{block}{GCU}
        \begin{enumerate}
            \item Study week 1 and 2 of the GCULearn content.
            \item Week 2 GCULearn Quiz \textbf{(Summative)}.
        \end{enumerate}
    \end{block}
\end{frame}

\begin{frame}{Tasks \& Deliverables}
    \begin{block}{ALU}
        \begin{enumerate}
            \item Implement a Stack ADT.
            \item Implement a Stack data structure.
            \item Use the Stack implementation to implement one of the following:
            \begin{enumerate}
                \item Prefix to Postfix conversion
                \item Postfix calculator
                \item Towers of Hanoi algorithm
            \end{enumerate}
        \end{enumerate}
        Deadline: Tuesday 04/06/2018 @ 16hr (Hard).
    \end{block}
\end{frame}

\begin{frame}{Assessment}
    \begin{block}{ALU}
        \begin{itemize}
            \item This is individual work, no plagiarism, no collusion.
            \item You may be asked to perform a Kata of your code in the compulsory lab class.
            \item Speak natural language, not Java (or any other programming language)
            \item Work will assessed with respect to the learning outcomes, no grades will be given.
        \end{itemize}
    \end{block}
\end{frame}

\begin{frame}{Feedback}
    \begin{block}{ALU}
        \begin{itemize}
            \item Individual feedback will be given by 18/06/2018.
            \item Feedback will cover code quality, algorithmic logic, efficiency and so on.
            \item General feedback will be given in class on 11/06/2018.
        \end{itemize}
    \end{block}
\end{frame}

\end{document}
